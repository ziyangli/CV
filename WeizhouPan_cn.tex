%% start of file `template.tex'.
%% Copyright 2006-2010 Xavier Danaux (xdanaux@gmail.com).
%
% This work may be distributed and/or modified under the
% conditions of the LaTeX Project Public License version 1.3c,
% available at http://www.latex-project.org/lppl/.

% Version: 20110122-4


\documentclass[11pt,a4paper,nolmodern]{moderncv}

\usepackage{WeizhouPan}

\usepackage[english]{babel}
\linespread{0.9}

\setCJKmainfont[BoldFont={Adobe Heiti Std},ItalicFont={Adobe Kaiti
  Std}]{Adobe Song Std}
\setCJKsansfont{YaHei Consolas Hybrid}
\setCJKmonofont{Adobe Kaiti Std}

\title{Weizhou Pan, 硕士研究生}
\address{计算机学院}{华南师范大学}
\extrainfo{\octocat~\url{http://github.com/wzpan}\\
  \linkedin~\httplink{www.linkedin.com/in/hahack}%
}
%\myquote{\href{www.hahack.com}{blog: www.hahack.com}}{}

%\extrainfo{\octocat~\url{http://github.com/wzpan}}

%\nopagenumbers{}                             % uncomment to suppress automatic page numbering for CVs longer than one page
%----------------------------------------------------------------------------------
%            content
%----------------------------------------------------------------------------------

\begin{document}
\hyphenpenalty=10000
\maketitle

\vspace{-2em}

\section{\hei 技能}

\cvcomputer{编程语言}{C/C++, Python, Shell/Bash, Matlab}
           {操作系统}{Linux, Windows}
\cvcomputer{英语能力}{CET6(513)}
           {开发工具}{Emacs, Subl, Git, Makefile, gdb}

\section{\hei 项目经验}

% Center labels and use "Since"
%\tltextstart[base]{\scriptsize}
%\tltextend[base]{\scriptsize}
%\tlsince{Since~}

\subsection{\hei 科研项目}

\tlcventry{2012}{2013}{\href{http://www.siat.ac.cn/}{中国科学院深圳先进技术研究
    院}}{客座学生}{}{}
{
  \begin{tightitemize}%
 \item 与\href{http://web.siat.ac.cn/~baoquan/}{陈宝权教授},
   \href{http://www.math.tau.ac.il/~dcor/}{Daniel Cohen-Or教授},
   \href{http://graphics.uni-konstanz.de/mitarbeiter/deussen.php}{Oliver
     Deaussen教授}等合作科研;
 \item 研究材质演变、形状检索、运动数据聚类等图形学课题;
 \item 开发二维百度地图控件和离线地图数据加密模块;
 \item 获得中国科学院客座学生奖学金.
 \end{tightitemize}}

\tlcventry{2011}{2012}{手势识别及红外感应在虚拟手术场景人机交互的研究和应用}{核心成员}{}{}%
  {
\begin{itemize}
 \item \textbf{课题性质}:广东省产学研结合项目子课题.
 \item \textbf{研究内容}:搭建了一个虚拟现实手术环境:采用红外感应技术实现了一支仿真手术刀,并用基于BP
   神经网络的手势识别技术漫游场景;
 \item \textbf{个人职责}:研究红外感应技术,基于 BP 人工神经网络的手势识别;
 \item \textbf{项目成果}:两篇核心期刊论文;“挑战杯”科技学术竞赛全国银奖,广东省特
   等奖;
\end{itemize}}

\tlcventry{2009}{2010}{基于 MLTS 技术的仿生歌唱系统}{负责人}{}{}%
  {
\begin{itemize}
\item \textbf{课题性质}:国家级大学生创新性实验计划项目.
\item \textbf{研究内容}: 设计了一种计算机学唱歌曲的系统:运用敲击定位法定
  位发音时刻,然后利用小波变换和快速傅里叶变换分析音调,最后采用语音合成技术输出声音.
\item \textbf{个人职责}:系统主要开发者,论文撰写人.
 \item \textbf{项目成果}:核心期刊论文.
\end{itemize}}


\section{\hei 论文和专利}
{
  \tldatelabelcventry{2013}{2013.4}{沉浸式手术环境的研究与实现}{计算机仿
    真}{第二作者}{已录用}{}
  \setlength{\parskip}{-20pt}
  \tldatelabelcventry{2013}{2013.2}{红外光虚拟手术刀的研究与实现}{计算机
    系统应用}{第一作者}{已录用}{}
  \tldatelabelcventry{2012}{2012.4}{一种基于小波和快速傅里叶变换的学习型歌唱系统
  }{计算机工程与应用}{第一作者}{已发表}{}
  \tldatelabelcventry{2013}{2013.8}{一种同步播放多媒体文件的方法及系统}{发
    明}{第一申请人}{申请中}{}
  \tldatelabelcventry{2012}{2012.12}{基于Clifford代数在十二方向上三维分割
    腹部血管的应用实例}{发明}{合作申请人}{申请中}{}  
}

\section{\hei 奖项}

% Restore normal labels
%\tltext{\scriptsize}

{
\tldatelabelcventry{2011}{2011.10}{第十二届“挑战杯”全国大学生课外学术科技
  作品竞赛银奖}{国家级}{}{}{}

\setlength{\parskip}{-20pt}
  
\tldatelabelcventry{2012}{2012.6}{华南师范大学创新奖学金一等奖}{校级}{}{}{}



\tldatelabelcventry{2011}{2011.4}{华南师范大学一等奖学金 $\times$ 2}{校级}{}{}{}

\tldatelabelcventry{2010}{2010.7}{华南师范大学二等奖学金 $\times$ 2}{校级}{}{}{}

}

\end{document}
