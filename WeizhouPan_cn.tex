%% start of file `template.tex'.
%% Copyright 2006-2010 Xavier Danaux (xdanaux@gmail.com).
%
% This work may be distributed and/or modified under the
% conditions of the LaTeX Project Public License version 1.3c,
% available at http://www.latex-project.org/lppl/.

% Version: 20110122-4


\documentclass[11pt,a4paper,nolmodern]{moderncv}

\usepackage{WeizhouPan}

\moderncvtheme[blue]{classic}

\usepackage[english]{babel}
\linespread{0.8}

\setCJKmainfont[BoldFont={Adobe Heiti Std},ItalicFont={Adobe Kaiti
  Std}]{Adobe Song Std}
\setCJKsansfont{YaHei Consolas Hybrid}
\setCJKmonofont{Adobe Kaiti Std}

\title{Weizhou Pan, 硕士研究生}
\address{RM 504~(510631)}{华南师范大学~计算机学院}
\extrainfo{\octocat~\url{http://github.com/wzpan}\\
  \linkedin~\httplink{www.linkedin.com/in/hahack}%
}
\myquote{Keep a simple and stupid mind.}{}

%\extrainfo{\octocat~\url{http://github.com/wzpan}}

%\nopagenumbers{}                             % uncomment to suppress automatic page numbering for CVs longer than one page
%----------------------------------------------------------------------------------
%            content
%----------------------------------------------------------------------------------
\begin{document}

\hyphenpenalty=10000
\maketitle

\section{技能}

\subsection{系统}
\cvline{操作系统}{GNU/Linux(ArchLinux, Ubuntu, RedHat, Linux Deepin, Fedora), Windows}
\cvline{桌面环境}{Awesome, xfce, KDE, Gnome2, Gnome2, Gnome3, Unity, Cinnamon, Mate}

\subsection{开发}
\cvcomputer{语言}{Python, Shell/Bash, C, Java, Pascal, Matlab}
	   {格式}{Markdown, Org-mode, Textile, reStructuredText, XML, YAML/JSon}

\cvcomputer{Web}{HTML, CSS, Mustache, Ruhoh, Wordpress, Twitter-Bootstrap, Web.py}
           {数据库}{MySQL, SQL Server}

\cvcomputer{编辑器}{Emacs, Vim, Subl, Visual Studio, Eclipse}
           {文档}{Doxygen, CHM}
           
\cvcomputer{方法}{设计模式, 面向对象, 测试驱动}
           {版本控制}{Git, SVN, GitHub, Google Code}


\subsection{工具}
\cvcomputer{办公}{Microsoft Office, WPS, OpenOffice/LibreOffice}
           {图形}{TikZ, Graphviz, Ditta, Gimp, Inkscape, Photoshop, Fireworks, CorelDraw, Visio, yEd, Calligra Flow}

\cvcomputer{效率}{Mind Map, GTD, Evernote, Dropbox, Org-mode, AutoHotKey, Snippets, Gist}
           {Wikis}{MoinMoin, MediaWiki, Github Wiki}
\cvcomputer{排版}{\TeX{}, \LaTeX{}, \XeTeX{}}
           {键盘布局}{QWERTY, Dvorak}

%\devnotes{Developer}{Contributor}

\section{履历}
\subsection{科研实践}

% Center labels and use "Since"
%\tltextstart[base]{\scriptsize}
%\tltextend[base]{\scriptsize}
%\tlsince{Since~}

\tlcventry{2012}{2013}{\href{http://www.siat.ac.cn/}{中国科学院深圳先进技术研究
    院}}{客座学生}{}{}
{
\begin{tightitemize}%
 \item 与 \href{http://web.siat.ac.cn/~baoquan/}{陈宝权教授},
   \href{http://www.math.tau.ac.il/~dcor/}{Daniel Cohen-Or教授},
   \href{http://graphics.uni-konstanz.de/mitarbeiter/deussen.php}{Oliver
     Deaussen教授}, \href{http://www.idav.ucdavis.edu/~asharf/}{Andrei Sharf教授} 等有着
   高效合作。
 \item 为 \href{http://web.siat.ac.cn/~yangyan/}{李扬彦} 博士的科研成果制作展示
   视频。
 \item 与 \href{http://web.siat.ac.cn/~kexie}{谢科博士}以及
   \href{http://www.idav.ucdavis.edu/~asharf/}{Andrei Sharf}教授合作论文。
 \item 为可视计算中心搭建一个
   \href{http://vcc.siat.ac.cn/w/index.php/Main_Page}{Wiki}系统.   
 \item 连续获得两个季度的客座学生奖学金。
 \end{tightitemize}}

\tlcventry{2011}{2012}{手势识别及红外感应在虚拟手术场景人机交互的研究和应用}{课
  题成员}{}{}%
  {
\begin{itemize}
 \item 合作完成并发表两篇论文。
 \item 负责基于任天堂 Wii Remote 的红外感应技术,虚拟手术刀的建模。
 \item 成果获得2012年华南师范大学软件设计大赛一等奖,2011年第十二届“挑战杯”全
   国大学生课外学术科技作品竞赛银奖等奖项。
\end{itemize}}

\tlcventry{2010}{2011}{\href{http://www.siat.ac.cn/}{中国科学院深圳先
    进技术研究院}}{客座学生}{}{}
{
\begin{tightitemize}%
 \item 开发基于QT和JavaScript二维百度地图控件;
 \item 开发基于TEA加密算法离线地图加密包;
 \item 百维达公司HR助理;
 \item 文献调研。
 \end{tightitemize}}

\tlcventry{2010}{2011}{招聘信息垂直搜索系统}{课题成员}{}{}%
  {
\begin{itemize}
 \item 基于Heritrix编写了一个网络爬虫。
\end{itemize}}

\tlcventry{2009}{2010}{基于 MLTS 技术的仿生歌唱系统}{负责人}{}{}%
  {
\begin{itemize}
 \item 发表一篇核心期刊论文。
 \item 成果获得华南师范大学软件设计大赛二等奖。
\end{itemize}}

\subsection{社会实践}

\tlcventry{2011}{0}{\href{http://www.gzlug.org}{广州Linux用户组(GZLUG)}}{成员}{}{}{
\begin{tightitemize}%
 \item \href{http://www.gzlug.org/?p=641}{2012 广州地区软件自由日庆祝活动}组织者;
 \item \href{https://github.com/gzlug}{GZLUG Github组织}内容提交者;
 \item GZLUG 邮件列表活跃用户.
\end{tightitemize}}

\tlcventry{2008}{2009}{华南师范大学校学生会网络部}{部长}{}{}
{
  \begin{tightitemize}%
    \item 校学生会第32届第3任网络部部长;
    \item 年底被推选为校学生会系统“优秀学生干部”。
 \end{tightitemize}}

\tlcventry{2007}{2008}{华南师范大学计算机学院07级5班}{班长}{}{}
{
  \begin{tightitemize}%
    \item 亲自策划并带头组织“班级文化节”;
    \item 年底被推选为校学生会系统“优秀学生干部”。
    \end{tightitemize}}

\tlcventry{2004}{2006}{陆丰市秋水报}{电脑编辑}{}{}
{
  \begin{tightitemize}%
    \item 负责电脑编辑工作,为《秋水》排版了4期报纸;
    \item 年底被评为秋水第三届“优秀编委”。
    \end{tightitemize}}

\subsection{奖项}

% Restore normal labels
%\tltext{\scriptsize}

{
\tldatelabelcventry{2012}{June 2012}{华南师范大学创新奖学金一等奖}{校级}{}{}{}

\setlength{\parskip}{-20pt}

\tldatelabelcventry{2011}{October 2011}{第十二届“挑战杯”全国大学生课外学术科技
  作品竞赛银奖}{国家级}{}{}{}

\tldatelabelcventry{2011}{June 2011}{华南师范大学优秀毕业生}{校级}{}{}{}

\tldatelabelcventry{2011}{May 2011}{华南师范大学优秀本科毕业论文}{校级}{}{}{}

\tldatelabelcventry{2011}{May 2011}{第十一届“挑战杯”广东大学生课外学术科技作品
  竞赛特等奖}{省级}{}{}{}

\tldatelabelcventry{2011}{April 2011}{华南师范大学一等奖学金}{校级}{}{}{}

\tldatelabelcventry{2010}{July 2010}{华南师范大学二等奖学金}{校级}{}{}{}

\tldatelabelcventry{2009}{October 2009}{华南师范大学一等奖学金}{校级}{}{}{}

\tldatelabelcventry{2008}{October 2008}{华南师范大学二等奖学金}{校级}{}{}{}

}

\section{教育背景}


\tlcventry{2011}{0}{\href{http://www.scnu.edu.cn}{华南师范大
    学}~\href{http://portal.scnu.edu.cn/wps/portal/cs}{计算机学院}}{研究生}{}{}{华南师范大
  学计算机学院2011届计算机应用技术硕士研究生}

\tlcventry{2010}{2011}{\href{http://www.siat.ac.cn/}{中国科学院深圳先进
    技术研究院}~\href{http://vcc.siat.ac.cn/}{可视计算中心}}{客座学生}{}{}{中国科学院深圳先
  进技术研究院可视计算中心客座学生}

\tlcventry{2007}{2011}{\href{http://www.scnu.edu.cn}{华南师范大
    学}~\href{http://portal.scnu.edu.cn/wps/portal/cs}{计算机学院}}{本科生}{}{}{华南师范
  大学计算机学院2007届网络工程系学生}


\section{英语水平}
\cvlanguage{英语6级}{513}{听力: 156; 阅读: 199; 综合技能: 64;写作: 94}



\end{document}
